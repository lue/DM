\section{Annihilation rate}

\subsection{Preliminaries}

The average rate of DM annihilations at redshift $z$ is proportional to the, $\langle \rho^2 \rangle$, the average of DM density squared. To calculate this value one needs to integrate over the volume. Unfortunatelly, the simulations of large scale structure do not have high enough resolution to pick up all substructure. Therefore we use approximate method based on converging the halos mass function and concentration relations.


For the halo mass profile we adopt standard NFW \cite{1997ApJ...490..493N}. 



\subsection{Calculactions}

Instead of going directly to the actual calculation of the annihilation rate, we make an intermediate step. We reveal an underling global probability density function (PDF) of DM overdensity. The volume weighted PDF:

\begin{equation}
\dfrac{dV}{d\delta}(z)=\mathrm{PDF}_V (\delta, z) = \int_{M_\mathrm{min}}^\infty \dfrac{dn}{dM}(M,z) dM \int_{r_{\delta-\Delta/2}}^{r_{\delta+\Delta/2}} 4\pi r^2 dr / \Delta
\end{equation}

The rate of annihilations can be derived from PDF:



We doublecheck that our results concistent with the usual way to estimate annihilation rate is through functions $B$ and $F$...

