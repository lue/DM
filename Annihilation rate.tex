\section{Annihilation rate}

\subsection{Preliminaries}

The average rate of DM annihilations at redshift $z$ is proportional to the, $\langle \rho^2 \rangle$, the average of DM density squared. To calculate this value one needs to integrate over the volume. Unfortunatelly, the simulations of large scale structure do not have high enough resolution to pick up all substructure. Therefore we use approximate method based on converging the halos mass function and concentration relations.

The halo mass function provides us with the information about abundances of halos in wide range of masses at high redshifts. Here we adopt two models: the analytical one based on ellipsoidal collapse \cite{Sheth_2001} and extension of the model by \cite{Tinker_2008} to high redshifts by \cite{Behroozi_2013}. The latter model is based on the fits to the numerical simulations. It explores only high masses, while low mass halos can be important ( ... ). We use this package for calculating HM functions \cite{Murray_2013}.

Next ingridient is the model for concentrations. We adopt the model by \cite{2014arXiv1407.4730D}, which provides superior fits. Again, since the model is based on simulation, only masses higher $~10^{10}\;M_\odot$ are probed. However, this concentration fits are in fair agreement with the simulations of micro-mass halos at high (~30) redshifts (see Figure 8 of \cite{2014arXiv1407.4730D}), which makes us optimistic about using it at lower masses and high redshifts.

Even though the average concentration for a halo of given mass is well defined, the scatter is not negligible. Therefore, we also integrate over the concentrations.

For the halo mass profile we adopt standard NFW \cite{1997ApJ...490..493N}. 

\subsection{Calculactions}
We are going to make an intermediate step befoire actual calculation of the annihilation rate. We will reveal an underling global probability density function (PDF) of DM overdensity.

\begin{equation}
\int_{M_\mathrm{min}}^\infty \dfrac{dn}{dM}(M,z) dM \int_0^{r_{200c}} \rho^2(M, r)*4\pi*r^2 dr
\end{equation}


