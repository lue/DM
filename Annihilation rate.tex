\section{Annihilation rate}

The average rate of DM annihilations at redshift $z$ is proportional to the, $\langle \rho^2 \rangle$, the average of DM density squared. To calculate this value one needs to integrate over the volume. Unfortunatelly, the simulations of large scale structure do not have high enough resolution to pick up all substructure. Therefore we use approximate method based on converging the halos mass function and concentration relations.

The halo mass function provides us with the information about abundances of halos in wide range of masses at high redshifts. Here we adopt two models: the analytical one based on ellipsoidal collapse \cite{Sheth_2001} and extension of the model by \cite{Tinker_2008} to high redshifts by \cite{Behroozi_2013}. The latter model is based on the fits to the numerical simulations. It explores only high masses, while low mass halos can be important ( ... ). We use this package for calculating HM functions \cite{Murray_2013}.

