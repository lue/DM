The byproducts of Dark Matter (DM) annihilation may effectively interact with the IGM and significantly contribute to the hydrogen ionization. In order to accurately compute the number of annihilations throughout the history of the universe, we adopt the latest results from structure formation studies. We study the effect of the remaining free parameters by subdividing their possible values into three groups Pessimistic/Realistic/Optimistic. We conclude that the mechanism of producing ionization photons by the DM annihilation very likely is only complimentary to the stars and is not efficient enough to reionize the universe by itself. However, the DM starts to efficiently annihilate with the formation of the first micro-halos at z~100, which is much earlier compared to the stars that are born at z~15-20. Therefore the DM annihilation stretches the reionization to higher redshifts. This prolonged reionization may lead to the increased optical depth of CMB, which potentially can be observed.
  