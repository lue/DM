The byproducts of Dark Matter (DM) annihilation may effectively interact with the IGM and significantly contribute to the hydrogen ionization. In order to accurately compute the number of annihilations throughout the history of the universe, we adopt the latest results from structure formation studies. We explore the effect of the remaining free parameters by subdividing their possible values into three groups Pessimistic/Realistic/Optimistic. The DM annihilation is not capable to ionize the universe by itself; thus, it play only complimentary role to the stars and quasars. However, the DM starts to efficiently annihilate with the formation of the first micro-halos at $z\sim100-200$, which is much earlier compared to the stars that are born at $z\sim15-20$. Therefore, the DM annihilation stretches the reionization to higher redshifts. The prolongation of reionization leads to the increased optical depth of CMB, which potentially can be observed.