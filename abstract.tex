The byproducts of Dark Matter annihilation may significantly contribute to the hydrogen ionization. We limit ourselves to the family of DM models and focus on the number of annihilations. In order to accurately compute the number of annihilations throughout the history of the universe, we adopt the latest results from structure formation studies in order to decrease the number of uncert. We study the effect of the remaining free parameters by subdividing their possible values into three groups Pessimistic/Realistic/Optimistic. We conclude that the mechanism of producing ionization photons by the DM annihilation very likely is only complimentary to stars and is not efficient enough to reionize the universe by itself. However, it starts to work much earlier whith the formation of the first micro-halos at z~100, while stars are born later at z~15-20, therefore the reionization streches to higher redshifts. This prolonged reionization may lead to increased opticall depth of CMB.
  