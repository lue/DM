\section{Results}
\label{sec:results}

In this section we present results of of the actual calculations. We compare two quantities. First is the global boosting factor as a function of redshift. It is independent from the ionization history and DM annihilation properties. Therefore it only quantifies the slumpiness of DM. Second is the ionization fraction of the universe as if it is ionized solely by products of DM annihilation. Later we will combine DM contribution with stars.

\section{discussion}

We have shown that the DM annihilation can potentially alter the very early stages of reionizaiton. It starts to produce ionizing photons well before stars and therefore effectively contribute into the ionizing fraction. 

However, the optical depth $\tau$ is an additive scalar parameter, therefore we the effect from stars and the DM annihilation cannot be separated without additional information. Currently, the star and galaxy formation is very uncertain especially at high redshifts.

The 21cm data is a very powerful that will allow us to constraint the contribution from stars, but that would require assumption on the model of reionization \cite{2015arXiv150908463L}. The DM annihilation does not accounted for into the classical approach.