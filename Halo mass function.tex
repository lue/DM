\subsection{Halo mass function}

As we will present in latter sections, low mass halos are main contributors to the total annihilation rate. Therefore our main requirement to the halo mass function is its ability to work at wide range of masses (up to $M \approx 10^{-12}M_\odot$) and at high redshifts (up to $z \approx 100$). 

There are two types of halo mass function models -- fully analytical based on collapse models and fits to the numerical simulations. Most of the fits to the numerical simulations like, \citet{Tinker_2008} an its extension to higher redshifts by \citet{Behroozi_2013}, are tuned only for halos $10^{8}M_\odot < M < 10^{15}M_\odot$. Low mass end can be tuned with special simulations like \cite{Diemand_2005}. In \cite{Schneider_2013}, the authors predict the mass function, which fits \cite{Diemand_2005}. The low mass end behaves as $M^{-1}$ and therefore can be modeled with a simple Press-Schecter mass function \citep{Press_1974}

At the moment we adopt only standard \citet{Press_1974} model with a cut-off as described in \cite{Diemand_2005}. We chose to the following cut-off masses, $M_{crit}$: $10^{-6}M_\odot$

Other references to be included:
ellipsoidal collapse \citep{Sheth_2001}; package for calculating HM functions \cite{Murray_2013}.

