\subsection{Halo mass function}

As we will present in latter sections, low mass halos are main contributors to the total annihilatoin rate. Therefore our main requirement to the halo mass function is its ability to work at wide range of masses (up to $M \approx 10^{-12}M_\odot$) and at high redshifts (up to $z \approx 100$). 

There are two types of halo mass function models -- fully analytical based on collapse models and fits to the numerical simulations.

Most of fits to the fits to the numerical simulations like, \citet{Tinker_2008} an its extension to higher redshifts by \citet{Behroozi_2013}, are tuned only for halos $10^{8}M_\odot < M < 10^{15}M_\odot$. Low mass end can be tuned with special simulations like \cite{Diemand_2005}. In \cite{Schneider_2013}, the authors predict the mass function, which fits \cite{Diemand_2005} MF!!!

Analytical models of spherical collapse \citep{Press_1974} and ellipsoidal collapse \citep{Sheth_2001} by definition work on any scale and at all redshifts.

At the moment we adopt only standard \citet{Press_1974} model. Laer we will probably include other models... We use this package for calculating HM functions \cite{Murray_2013} (which actually does not work properly...).
