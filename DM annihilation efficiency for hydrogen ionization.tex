\section{Ionization efficiency by DM annihilation}
\label{sec:DMspectrum}

\cite{Slatyer_2009}
\cite{H_tsi_2009}
\cite{Evoli_2013}

\subsection{Electrons}

High energetic electron dissipates its energy into ionization, excitation and heat of IGM. This process has been well studied for energies below 10 keV in \cite{Shull_1979,Shull_1985,Dalgarno_1999,Furlanetto_2010}. In \cite{Vald_s_2010} the authors consider higher energies but neglect time dependence. The process off delayed effect of energetic particles is shown, for example, in \cite{2015arXiv150603812S} where the effect of DM annihilation on post-recombination era is studied. The fits provided there are not suitable for our study, because they do not account for dramatic ionization fraction changes during reionization.

Therefore we perform our own calculations of electron energy deposition. For energies below few keV we adopt the results from previous studies \cite{Shull_1985,Furlanetto_2010}. Higher energy electron will quickly decelerate through Inverse Compton on CMB photons until it will enter the regime where collisional ionization and excitations become important. This energy threshold depends on ambient ionization fraction and CMB photon density, i.e. redshift.

The spectrum of IC photons from decelerating charge from one energy to another is calculated numerically for each individual case.

Here we will describe the calculation of the number of ionizations per one anihilation.

According the Equation 12 of \cite{Belikov_2009}, the number of ionization per photon of energy $E_\gamma$:
\begin{equation}
N_\mathrm{ion}(E_\gamma) \approx 2.4\times10^7\dfrac{E_\gamma}{1\,\mathrm{GeV}}
\end{equation}

On the other hand, in the \cite{Belikov_2010} (the last sentence of appendix) the photon spectrum of IC is normalaized to the total energy of electron/positron pair from annihilation. If I understand correctly, it leads to the conclusion that the number of ionization per annihilation is roughly a constant fraction (1/3) of $m_X/E_i$, where $m_X$ is the mass of DM particle and $E_i$ is the ionization threshold of hydrogen.


The process which is competing with reionization is recombination. Here we consider a region with overdensity $\delta$ and ionization fraction $x_i$. The recombination rate is:
\begin{equation}
R = \bar{n}_b^2 (1+\delta)^2 x_i^2 \alpha_H,
\end{equation}
where $\alpha_H$ is a recombination coefficient
  
  