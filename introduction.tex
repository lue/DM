\section{Introduction}

The history of cosmic reionization remain uncertain because of unsufficient observations. Current observations allow us to put determine the very end of reionization and very broad constraints on its duration. Therefore, the physics of reionizaiton is still broadly unknown.

The main quest of reionization is to determine the sources of ionizing photons. The most accepted sources for Hydrogen ionization are stars within galaxies, and quasars for Helium reinization. Recent observation allow ... to argue in favor of quasar activity during Hydrogen reionization too.

The upcoming observations by JWST and 21cm experiments will make possible to put much stricter constraints on the star formation rate at high redshifts, and therefore fix the galaxy formation.

This uncertainity gives freedom to include other sources of ionizing photons. For instance X-ray binaries. The alternative sources will behave different from stars by altering the reionization history. 

In this scope, the DM annihilation is an interesting process. The products of DM annihilation may affect the IGM and therefore change the global ionization and heat history. The measured recombination history with CMB observation by Plank allows to put a significant constraints on the DM particle properties.

In order to estimate the effect of DM annihilation on the IGM during Dark Ages and the epoch of reionization, we need to take into account the following three components:
\begin{enumerate}
\item The boosting factor of DM  
\item The products of DM annihilation. We leave out scope of this paper the hadronization ... We use the spectrum of 
\item
\end{enumerate}
In \S\ref{sec:preliminaries} we discuss general questions and problems of calculations of dark matter annihilation.

In \S\ref{sec:origins_of_clumpiness} we describe each effect, which contribute to the clumpiness of DM and increases global boosting factor.

In \S\ref{sec:DMspectrum} we discuss the efficiency of hydrogen ionization by DM annihilation products.

In Appendix we examine whether or not we capture all regions which significantly contribute to the total annihilation rate. In other words we check that annihilations in voids are insignificant.
