\section{Introduction}

Because of the limited observations, the physics of cosmic reionization remains very uncertain. Currently available observations allow us to probe the very end of reionization (Ly$\alpha$ forest) and to put very broad constraints on its duration (CMB polarization). The upcoming observations by JWST and 21cm experiments will make possible to observe in details high redshifts and to put much stricter constraints on the reionization. However, till then alternative scenarios of the reionizaiton are not excluded.

The main quest of reionization is to determine the sources of ionizing photons. The most widely accepted sources for Hydrogen ionization are the stars within galaxies, and the quasars for Helium reionization. Recent observations allow \cite{2015arXiv150707678M} to argue in favor of quasar activity during Hydrogen reionization too. Some other theories consider other sources, for instance, X-ray binaries \cite{Fialkov_2014}. In order to distinguish between sources we have to find observables that are sensitive to the specifics of these sources.

In this scope, the DM annihilation is an interesting process. The products of DM annihilation may affect the IGM, and therefore change the global ionization and heat history. The detailed studies of the epoch of recombination together with CMB observation allows \citet{2015arXiv150603811S} to put significant constraints on the DM particle properties. Therefore it motivates us to perform a similar analysis but now using the epoch of reionization. 

In contrast to some previous studies \cite{2009JCAP...10..009C, 2009PhRvD..80c5007B} we do not assume that DM is the only or the main source of ionizations. Guided by previous studies \cite{H_tsi_2009}, we expect the DM to be a minor contributor to the total number of ionizing photons. However, in contrast to the stars that start to form at $z\sim 15-20$, the DM begins to annihilate much earlier at $z\sim100-200$ when the first DM micro-halos start to collapse. We show that by ionizing the universe to a few percent level in the interval between redshift $\sim20$ and $\sim200$ may significantly affect the global optical depth, $\tau$, measured with CMB. We show that combined with future 21cm the reionization can provide another constraint on the DM particle properties.

The proper estimation of the effect of DM annihilation on the IGM during Dark Ages and the epoch of reionization requires to take into account the following three components.

Firstly, the total rate of DM annihilation should be evaluated. It is determined by boosting factor, which quantifies the amount of structure. In \S\ref{sec:boosting} we describe each effect, which contributes to the clumpiness of the DM and increases the global boosting factor.

Secondly, the products of DM annihilation. We limit ourself only to $b\bar{b}$...
\textbf{AK: Dan, please, add something here.} Discussed in \S\ref{sec:DM}.

Lastly, the interaction of DM annihilation products with the IGM is not limited by ionization; therefore, the interaction of energetic electrons with the IGM has to be modeled. In \S\ref{sec:DMspectrum} we discuss the efficiency of hydrogen ionization by DM annihilation products.

In Appendix we provide fits for the boosting factor that can be used in other studies.