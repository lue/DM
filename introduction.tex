\section{Introduction}

In \S\ref{sec:preliminaries} we discuss general questions and problems of calculations of dark matter annihilation.

In \S\ref{sec:origins_of_clumpiness} we describe each effect, which contribute to the clumpiness of DM and increases global boosting factor. For each parameter we adopt three values, which correspond to pessimistic, realistic and optimistic scenarios (we assume that more annihilations happen -- better the world we live in is). Probably, it is better to name them as conservative/realistic/... .

In Appendix we examine whether or not we capture all regions which significantly contribute to the total annihilation rate. In other words we check that annihilations in voids are insignificant.
