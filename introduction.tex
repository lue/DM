\section{Introduction}

The history of cosmic reionization remain mostly uncertain because of insufficient observations. Current observations allow us to determine the very end of reionization and ti put very broad constraints on its duration. Therefore, the physics of reionizaiton is still broadly unknown.

The main quest of reionization is to determine the sources of ionizing photons. The most accepted sources for Hydrogen ionization are stars within galaxies, and quasars for Helium reionization. Recent observation allow \cite{2015arXiv150707678M} to argue in favor of quasar activity during Hydrogen reionization too.

The upcoming observations by JWST and 21cm experiments will make possible to put much stricter constraints on the star formation rate at high redshifts. However, now the existing uncertainity gives freedom to consider other sources of ionizing photons. For instance X-ray binaries \cite{Fialkov_2014}. The alternative sources will behave different from stars by altering the reionization history. 

In this scope, the DM annihilation is an interesting process. The products of DM annihilation may affect the IGM and therefore change the global ionization and heat history. The measured recombination history with CMB observation by Plank allows \cite{2015arXiv150603811S} to put a significant constraints on the DM particle properties.

We perform a similar calculation but now for the reionization. In contrast to some previous studies \cite{2009JCAP...10..009C, 2009PhRvD..80c5007B} we do not assume that DM is the main source of reionization. We expect the DM to be a minor contributor to the total number of ionizing photons. In contrast to the stars that start to form at $z\sim 20-30$, the DM begins to annihilate much earlier at $z\sim100-200$ when the first DM micro-halos start to form. We show that by ionizing the universe to a few percent level in the interval between redshift $\sim20$ and $\sim200$ may significantly affect the global optical depth, $\tau$, measured with CMB. We show that combined with future 21cm the reionization can provide another constraint on DM particle properties.

In order to estimate the effect of DM annihilation on the IGM during Dark Ages and the epoch of reionization, we need to take into account the following three components:

Firstly, the total rate of DM annihilation should be determined. It is determined by boosting factor, which quantifies the amount of structure. In \S\ref{sec:origins_of_clumpiness} we describe each effect, which contributes to the clumpiness of the DM and increases global boosting factor.

Secondly, the products of DM annihilation. We limit ourself only to $b\bar{b}$...
\note{AK: Dan, please, add something here.}

Lastly, the interaction of DM annihilation products with the IGM is not limited by ionization; therefore, the interaction of energetic electrons with the IGM has to be modeled. In \S\ref{sec:DMspectrum} we discuss the efficiency of hydrogen ionization by DM annihilation products.

In \S\ref{sec:preliminaries} we discuss general questions and problems of calculations of dark matter annihilation.


In Appendix we examine whether or not we capture all regions which significantly contribute to the total annihilation rate. In other words we check that annihilations in voids are insignificant.
