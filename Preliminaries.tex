\section{Boosting factor}
\label{sec:boosting}

The rate of DM annihilations at redshift $z$ is proportional to the, $\langle n_{DM}^2 \rangle$, the average of DM number density squared. To calculate this value one needs to integrate over the whole volume. Unfortunatelly, the simulations of the large scale structure do not have high enough resolution to pick up all the substructure. Therefore, we use approximate method based on converging the halos mass function and halo density profiles.

In order to perform this calculation we will need various information regrading the DM halo profiles and the halo mass function. Even though the physics of the large scale structure is well studied, some of the parameters remain uncertain or unknown. This freedom in the parameter space may lead to different values of the boosting factor. We attempt to systematically approach this problem by classifying the parameters into three categories: \textit{Low},\textit{Medium}, and \textit{High}. Each parameter is discussed in a separate subsection; and the values for the parameters are presented in Table 1.

After establishing the range of parameters we explore the importance of each of them in \S ... 

\begin{table}
    \begin{tabular}{ c c c c }
    \textbf{Parameter description}                                    & \textbf{Low} & \textbf{Medium} & \textbf{High} \hline \hline \\
    Mass function cut-off (in $\log_{10}M_\odot$)        & -6 & -9 & -12 \\
    Scatter of concentrations (in $\sigma_{\log_{10}c}$)              & 0.10 & 0.15 & 0.20 \\
    Modified NFW profile for small halos (inner slope alpha) & $\alpha=1$ & \cite{Ishiyama_2014} &  \cite{Ishiyama_2014} \\
    Subhalo mass function & None & 0 & 0 \\
    Additional clumping due to caustics and non-spherical profiles & None & None & 2.0
\\    \end{tabular}
\end{table}
